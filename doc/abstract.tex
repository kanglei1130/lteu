LTE-U is a recently proposed wireless standard to
allow LTE to use unlicensed Wi-Fi spectrum. It is implemented
in the latest generation of handsets and small cells, and ready for
deployment. Several academic and industry studies have pointed
to various potential coexistence issues between LTE-U and Wi-Fi,
mainly centered around energy sensing and rate adaptation. 
In this paper, we present our findings from implementing 
an open LTE-U platform and evaluating the impact 
of LTE-U interference on existing Wi-Fi networks.
We find that various sensing and rate adaptation algorithms
implemented in different Wi-Fi firmwares show 
various sensitivities to LTE-U interference. 
For a single Wi-Fi link, 
LTE-U interference has limited impact on OpenWRT routers, while
DD-WRT and some commercial routers are very sensitive to 
LTE-U interference under different LTE-U settings. 
Qualcomm’s CSAT algorithm is currently
the only algorithm proposed to adjust duty cycle for general
coexistence scenarios when there are multiple links, and it has been
implemented in the first generation of LTE-U devices. We show
that CSAT design principles are at odds with CSMA, regardless
of parameter setting, and can cause severe unfairness. We
propose a novel algorithm, which we call reactive adaptation, that
significantly improves coexistence in several important scenarios.
In order to evaluate our
proposals, we develop an open source implementation of LTE-U
on a software defined radio platform and we use it to quantify our
claims by measuring effect on real-world Wi-Fi deployments. We
released our LTE-U system to the public to assist further studies.



%We further show that there are scenarios in which the main
%premise of LTE-U, not to sense the medium before transmitting,
%causes fundamental inefficiency that cannot be fully alleviated
%by any further protocol improvements. 

%In this paper, we focus on a novel aspect of the coexistence problem,
%which is the one of the fair medium access between CSMA and
%TDMA. Wi-Fi is a CSMA-based technology which automatically
%adapts its transmission rates to congestion through sensing and
%back-offs. LTE-U is a TDMA-based technology which does not
%sense the medium before transmitting and has no default way to
%adapt to congestion. 




