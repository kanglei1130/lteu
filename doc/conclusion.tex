
In this paper, we presented a thorough evaluation of the interference
caused by LTE-U to existing Wi-Fi networks. 
To achieve this, we have developed an  
open LTE-U programmable software defined radio platform, 
for which we made the entire source code publicly available as open source. 
We show that different sensing and rate adaptation algorithms implemented
in various firmwares, such as DD-WRT and OpenWRT, 
have different levels of LTE-U interference handling capabilities.
In our measurement, OpenWRT router shows the best 
existence capability with LTE-U. 
Our results demonstrate several deficiencies 
of the existing LTE-U protocol, including the incorrect duty cycle adaptation, 
lack of proper carrier sensing and negative interaction of LTE-U collisions on Wi-Fi rate adaptation in DD-WRT and commercial Wi-Fi routers.
These can severely degrade Wi-Fi performance leading to lack of fairness when the two technologies coexist. 
To remedy this, we propose solutions to each of the observed problems, and we show how these
solutions alleviate current LTE-U deficiencies both in controlled experiments and in the wild.


%Finally,  we also show that LTE-U causes a fundamental inefficiency in spectrum access due to the lack of sensing.
%Thus, even with all of our proposed solutions, we believe that, ultimately, 
%carrier sensing is the right way to enable fair 
%coexistence between LTE-U and Wi-Fi, especially given the imminent 
%deployment of LTE-U in unlicensed spectrum.

