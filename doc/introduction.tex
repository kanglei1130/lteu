Recent dramatic increase in mobile traffic has put the existing wireless and cellular networks under strain. 
Cellular vendors and operators are trying to leverage unlicensed bands to mitigate spectrum scarcity.
For example, Verizon has announced plans to deploy one such technology called LTE-U (LTE Unlicensed) in the United States.
LTE-U is based on the LTE standard and will aggregate both licensed spectrum and unlicensed spectrum for LTE traffic.
The signalling, control channel, uplink, and portion of downlink will use the licensed band as usual, but the bulk of downlink traffic will be transmitted over the 5GHz unlicensed bands.  

Today, the primary use of 5GHz unlicensed band is Wi-Fi.
Wi-Fi uses a medium-sharing mechanism known as listen-before-talk (LBT), where nodes sense and wait for free medium before each transmission.
However, LTE-U operates on a predetermined transmission schedule (ON-OFF duty cycle) without per-transmission LBT.
The lack of LBT will no doubt lead to collisions with ongoing Wi-Fi transmissions. 
LTE-U relies on Wi-Fi's loss recovery mechanisms through retransmission and Wi-Fi's LBT mechanism to avoid transmissions during LTE-U's ON periods.
Further, LTE-U's medium sharing strategy is unilateral.
To offer fair sharing, LTE-U reads the medium when it is not transmitting and adjusts its duty cycle based on a medium utilization (MU) estimation.
The basic idea is that, if medium utilization is low, Wi-Fi load must be small and LTE-U can increase its share of medium use by increasing duty cycle, and vice versa. 

%The coexistence between Wi-Fi and LTE-U is an academically interesting problem with severe practical implications.
%Fundamentally, this is a question about whether a CSMA-type protocol can effectively coexist with a TDMA protocol without LBT.
%More specifically, it is also a question of whether duty cycle adaptation based on inferred medium utilization values can replace per-packet LBT and still yield to a stable and fair sharing of medium.
%Answering these questions is fundamental to our understanding of the deep implications of running LTE-U on the same channel as Wi-Fi.

The coexistene between Wi-Fi and LTE-U has severe practical implications because LTE-U is now real and imminent.
%not just an academic exercise.
LTE-U specification V1.3~\cite{lteuforum_lteu} has been approved and published by the industrial LTE-U Forum in October 2015.
It has been included in the newest generation of Qualcomm cellular chipsets (Snapdragon X12) and their latest small cell reference designs (FSM9955)~\cite{x12}. 
Using this chipset, Samsung's next flagship phone, Galaxy S7, is now ``LTE-U capable''~\cite{s7}.
Samsung and SpiderCloud Wireless have started marketing LTE-U small cell products.
Verizon has announced plans for a large-scale roll-out in the United States, pending FCC approvals. 
Several interested parties are planning real-world trials in an effort to persuade FCC certification.
The potential fallout on tens of millions of 5GHz Wi-Fi devices will be huge, 
if coexistence effects are not properly understood before mass scale deployment. 



In this paper, we measure the impact of LTE-U interference on existing Wi-Fi networks.
As a start, we measure the throughput and latency of single link Wi-Fi when varing
the duty cycle, duty cycle, on time, the number of filled resource blocks 
of LTE-U frames. 
We conduct the same measurement on different Wi-Fi routers, 
i.e., OpenWRT, DD-WRT and commercial routers. 
Our extensive measurements show that OpenWRT is resistant to LTE-U
interference, while DD-WRT and some commercial routers
show different levels of vulnerabilities. 
Our observations indicate: 1. LTE-U should be better designed
so that most deployed Wi-Fi routers can be less affected; 
2. The sensing and rate adaptation algorithm of Wi-Fi routers
can be well defined and carefully implemented to 
be resistant to external interference. 
For genenral coexistence cases when there are multiple Wi-Fi links, 
duty-cycle adaptation algorithm is central for LTE and Wi-Fi coexistence.
However, there is only a single coarsely defined algorithm, 
CSAT~\cite{lteuforum_csat}, proposed for LTE-U to derive a fair medium share.
Surprisingly, CSAT has so far been evaluated only in a very specific, 
complex topology and with a synthetic workloads~\cite{qualcommpresentation}. 
In our measurement, we show that CSAT doesn't work well in a wide range of real-world situations, including an omnipresent hidden-terminal topology. 


Inspired by the coexistence mechanims (i.e., CTS-to-self) 
among various versions of Wi-Fi networks, 
i.e., 802.11a/b/g/n, 
we implement CTS-to-self in software defined radio
and demonstrate LTE-U can coexsit with various
Wi-Fi networks with the extra header overhead. 
We also propose a novel duty-cycle adaptation algorithm called {\em reactive adaptation}
for LTE-U to coexist with public Wi-Fi networks.
The key observation is that a medium utilization observed 
by a single node in a saturated network does not have to be high, 
as we show in Section~\ref{sec:csat}. 
Instead, we base our duty cycle adaptation on a relative change of 
the medium utilization observed as a reaction to changes in the duty cycle. 

In order to verify our findings, we first built an open source LTE-U prototype platform~\cite{OpenLTEU} and a reference implementation of the coexistence algorithms, including our own reactive adaptation algorithm. 
Then, we use the platform to run a series of LTE-U coexistence tests on real-world Wi-Fi deployments. 
We show that with the default LTE-U implementation, Wi-Fi almost never gets its fair share, in contrast to previous studies~\cite{qualcommpresentation} that focused on small lab deployments. 
%We also show that previously suggested remedies~\cite{google, cablelabs} -- improved rate adaptation and preamble sensing -- only slightly improve Wi-Fi's performance. 
We further show that in most cases reactive adaptation alleviates the coexistence issues. 

%However, we also observe that in a few cases LTE-U exhibits a {\em fundamental efficiency problem} that we discuss in Section~\ref{sec:ineff}.
%Without LBT, LTE-U transmissions can cause Wi-Fi performance to drop by as much as 30\%, which we show to be avoidable if Wi-Fi sensing were in place.
%This avoidable inefficiency proves that LBT should be the preferred way for any wireless technology to coexist with CSMA-type protocols.


In summary, this paper makes the following contributions:
\begin{itemize}
\item We developed an open LTE-U test platform~\cite{OpenLTEU} 
  and conduct systematic study for 
  the impact of LTE-U interference on Wi-Fi networks.
  Our LTE-U platform can run on multiple hardware platform such as 
    Sora \cite{sora} and BladeRF \cite{bladeRF}. 

\item We conduct the measurement on different Wi-Fi routers, 
  i.e., OpenWRT, DD-WRT and commercial routers,
  and show that different sensing and rate adaptation algorithms have 
    significant differences of sensitivities to LTE-U interference.

\item We analyze one of the critical coexistence issues of duty cycle adaptation and its impact on Wi-Fi performance. We show deficiencies of the currently proposed mechanism and we present a novel algorithm that significantly improves performance in many cases. 
We verify our findings in both controlled environment and in the wild real world.

\end{itemize} 


%\item We further show that in some cases LTE-U is fundamentally less efficient in spectrum use due to the lack of per-transmission LBT.  
%Wireless coexistence in unlicensed spectrum has always been an important problem.
%However, it has now become critical with the advent of new technologies that can potentially threaten the survival of one of the most successful wireless technologies so far -- Wi-Fi. 
%We hope that our study is one of the first step towards understanding the notion of fairness and building a propoper coexistence framework between CSMA and TDMA-based technologies in valuable unlicensed spectrum. 

In next section we discuss the basics of LTE-U and we introduce the coexistence problem. 
In Section~\ref{sec:platform}, we present the system component
and implementation details of our Open LTE-U platform.
In Section~\ref{sec:singlelink}, we measure the impact of LTE-U
interference on single Wi-Fi links.
In Section~\ref{sec:csat}, we explain CSAT, show its inefficiencies.
We present our reactive adaptation algorithm in Section~\ref{sec:solution}. 
%In Section~\ref{sec:ineff} we discuss the remaining fundamental coexistence issues due to lack of sensing. 
In Section~\ref{sec:eval}, we present our real-world coexistence evaluation, followed by related work, discussions and conclusions.

