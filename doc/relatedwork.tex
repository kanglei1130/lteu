\subsection{Related Work}

Despite the potentially huge real-world impact of LTE-U,
research studies examining Wi-Fi coexistence problems have been 
limited. This is mainly due to the lack of a publicly available LTE-U system.
The only result on a LTE-U running system is provided by Qualcomm,
which implements LTE-U in a chipset~\cite{qualcommpresentation,qualcommiccw}.
Qualcomm concludes that LTE-U is a better neighbor of Wi-Fi than Wi-Fi itself based on a set of controlled experiments.
However, its LTE-U system which contains a proprietary CSAT algorithm is not publicly available for researchers to reproduce the results or conduct a more extensive study.
On the other hand, Google~\cite{google}, Cablelabs~\cite{cablelabs} and NEC Lab~\cite{chai2016lte} raise several concerns on the coexistence problem 
based on an emulated LTE-U signal from test equipments or simulation.
The study presented in this paper extends existing results in a number of aspects.
First, we base our study on a complete and publicly available LTE-U system.
Results can be easily reproduced.
Second, we examine several Wi-Fi routers and show that the impacts of LTE-U
are different due to various sensing and rate adapatation algorithms. 
Third, we conduct an extensive study on all the major coexistence issues.
To our knowledge, this is the first study of CSAT and its properties. 
Fourth, we provide solutions for discovered coexistence issues and validate 
them through both controlled and in-the-wild experiments.

Besides results discussed above,
the problem of LTE in unlicensed band has also attracted simulation studies~\cite{simulation1,simulation2,simulation3}.
These studies usually focus on a certain aspect,
\textit{e.g.}, collision probability and delay~\cite{simulation1}.
Moreover, there are also several solutions proposed for the coexistence problem beyond medium access protocols.
For example,
inter-network coordination~\cite{solution2},
advanced physical layer technique~\cite{solution3},
spectrum access based on mathematical sharing model~\cite{solution1},
\textit{etc}.

\subsection{Discussion}

In the following, we discuss several 
issues related to the coexistence problem between LTE-U and Wi-Fi that have not been addressed in the previous sections. 

%% \textbf{Can CSMA solve the coexistence problem?}
%% Applying carrier sensing to LTE-U can help to avoid unnecessary collisions with Wi-Fi.
%% However, carrier sensing itself does not guarantee fair coexistence between LTE-U and Wi-Fi.
%% A CSAT-like scheduling algorithm is still in need for fair channel sharing,
%% unless the same parameter set of Wi-Fi (\textit{e.g.}, exponential back-offs, TXOP, \textit{etc.}) is applied in addition to carrier sensing.

\textbf{Can LAA solve the coexistence problem?}
The upcoming \textit{License Assisted Access} (LAA) will include listen-before-talk,
therefore is promising to overcome the inefficiency problem of LTE-U.
However, fairness will depend on the exact specification, and this remains to be confirmed in a separate study. 
%However, as discussed above, listen-before-talk does not guarantee fair coexistence.
%We do not know whether LAA will provide fair coexistence with Wi-Fi before the LAA specifications are finished.


\textbf{LTE-U/LTE-U coexistence?} While this paper focuses on 
LTE-U and Wi-Fi coexistence, we expect 
multiple LTE-U base stations on the same 5GHz channel should LTE-U become
mainstream. In such a case, LTE-U base stations  
 will impact each other significantly. 
As none of them performs listen-before-talk, 
the actual degradation could be arbitrarily severe.
%To mitigate the problem, multiple co-located LTE-U base staions should use non-overlapped 5GHz channels or synchronize their transmissions using out-of-band channels.

\textbf{Is 5GHz popular?}
According to ABI research, we have more than 10B wi-fi by early 2015
in which more than 50\% of Wi-Fi chip shipment is dual-band, \textit{i.e.}, 5GHz capable~\cite{wifimarket1},
and more than 18B chips will be shipped between 2015 to 2019~\cite{wifimarket2}.
Given that by 2019 we will have 30B Wi-Fi and at least 1/3 of them (10B) will be potentially 5GHz capable,
collocated LTE-U and Wi-Fi operating on the same 5GHz channel will be a common case that must be addressed.

\textbf{What about 802.11ac?}
802.11ac introduces two new channel width settings for Wi-Fi: 80MHz and 160MHz. While LTE-U operates on a 20MHz channel width, it is not clear what is the impact on a 80MHz or 160MHz channel. We have not used 802.11ac in our study, and this should be an important future work.
